% Chapter 1

\chapter{Introduction to nuclear reactions} % Main chapter title
In recent years, the study of light weakly bound nuclei  has not lost interest due to the successful development of experimental technology. 
In particular, it should be noted a significant achievement in the field of production of secondary radioactive beams. 
It is known that in light nuclei nucleons tend to group into clusters, the relative motion of which mainly determines the properties and characteristics of the being explored nuclei. 
The cluster structure of the ground and low-lying excited states of light exotic nuclei is one of the priority areas of both experimental and theoretical nuclear physics.


Exotic states in most cases have a rarefied structure and an increased size, which are reflected in rms radii. Such peculiar properties are manifested in the second excited state of the $^{12}$C 7.65 MeV nucleus (Jp = 0+), which in the framework of many well-known models has a cluster structure. 
A similar behavior is predicted for excited states in nuclei 11B and 13C, possibly also having Hoyle states.
 In the framework of the $\alpha$-condensate theory [4, 5], the radius of the Hoyle state is approximately 1.4–1.7 times the radius of the ground state, and the existence of states with a radius close to the radius of uranium nuclei is predicted in $^{12,13}$C and $^{11}$B nuclei.


In the framework of the cluster model, atomic nuclei can be formed from simple particles, such as ions: deuterium-2, tritium-3, hellium-3 alpha and others.
 As an example, $^{6}$He and $^6$Li nuclei can be used. 
 Based on the successful application of theoretical research within the framework of the hyperspherical harmonic model [6], as well as correlation gaussoids [7], it can be deduced that the $^{6}$He and 6Li nuclei are well described by three-body cluster models $\alpha$ + n + n and $\alpha$ + n + p, respectively.
The clustering effect is manifested in many cases in unstable nuclei located on the border of the stability of the nuclear map.
 But there are stable atomic nuclei that clearly show the cluster structure as 7Li, $^{9}$Be, $^{12}$C, and 16O. 
 The $^{9}$Be nucleus is of particular interest for research, since it is a stable, but at the same time weakly bound nucleus. 
 For example, the binding energy of one neutron Sn ($^{9}$Be) = 1.7 MeV in $^{9}$Be is less than the binding energy of one neutron Sn ($^{6}$He) = 1.9 MeV in the unstable $^{6}$He nucleus. 
 Moreover, the $^{9}$Be nucleus has a Borromian structure in which each pair combination in the triple $\alpha$ + $\alpha$ + n structure has no bound state.


The cluster configuration of the $^{9}$Be nucleus is also of interest in nuclear technology. In particular, this nucleus plays an important role in thermonuclear fusion. It is known that the larger the proton number in the atomic nucleus of the wall material, the more the material mixture is formed. In [8] calculations show that with a low proton number the $^{9}$Be nucleus is the most suitable material for the wall of thermonuclear devices. Moreover, if one take into account that the $^{9}$Be nucleus is easily decomposed by electrons and gamma particles into two fast alpha particles, then their kinetic energy contributes to the burning of fuel in the active zone.

It is widely acknowledged that 8Be and 5He nuclei do not exist in nature. 
The lifetime of these nuclei is very short -  approximately 10-20 seconds.
 However, the formation of new elements through these unstable nuclei exists. 
 In particular, the formation of the $^{9}$Be nucleus based on 8Be and 5He nuclei is strongly suppressed. 
 However, in [9] it was shown that under certain circumstances this synthesis is possible. 
% For example, at a temperature of T 109 K, the synthesis of the $^{9}$Be nucleus through the 5Не nucleus becomes dominant.
It is interesting to note the properties of neutron-rich beryllium isotopes. The addition of three neutrons to $^{9}$Be leads to the filling of the neutron p-shell. Along with other beryllium isotopes, the 12Be nucleus has a sufficient lifetime (21.3 ms) for registration and a binding energy of 3.17 MeV greater than the binding energy of the stable beryllium isotope. Therefore, the 12Be core is a remarkable subject of study within the framework of both the shell and cluster models. The study of the structure of radioactive atomic nuclei within the framework of the shell model is interesting because it allows one to take into account all degrees of freedom of nucleons, use realistic effective NN interactions and take into account the contribution of three-nucleon forces when describing the structure of the nucleus [10]. The model describes well the ground states p, sd shell nuclei, but not resonant excitations in the continuous spectrum. To study the state of the continuum, it is necessary to use theoretical approaches that correctly take into account the asymptotic behavior of wave functions at large distances.
Beryllium isotopes exhibit an exceptional $\alpha$-cluster structure — the ground and excited states form molecular structures with two $\alpha$-particles bound by additional neutrons. The configuration of this kind of light nuclei is manifested not only in the structure, but also in the mechanism of interaction with other nuclei. To a large extent, this is observed in the cross sections of the channels of the fusion and transfer reactions. It is also interesting to note that the clustering effect in light nuclei can manifest themselves during the interaction by means of exotic systems, such as 2n, 8Be, 5He, etc. For example, 8Be and 5He nuclei are associated with a breakup as an intermediate channel for a stable $^{9}$Be nucleus. In this case, it is important to note the works [9, 11],  which shows the path of the breakup of this nucleus. In this experimental work, it was proved that low-lying states of the $^{9}$Be nucleus up to an energy of 4 MeV have an n + 8Be configuration, and high-lying excited states from 4 MeV have a 5He + $\alpha$ structure.
In addition to studying the cluster structure in breakup reactions, it is interesting to note the interaction of deuterons with clustered nuclei [12]. In [12] conducted experiments to study the interaction of deuterons with $^{9}$Be nuclei. In particular, the nuclear reaction $^{9}$Be(d,a)7Li should be noted. For this nuclear reaction, different reaction mechanisms were evaluated: direct deuteron transfer, contribution of evaporation residues, and transfer of the 5He heavy cluster. The contribution of the mechanism through the compound nucleus, calculated in the framework of the statistical method, has a small contribution for this reaction at a laboratory energy of 7 MeV. It turned out that a large contribution to the cross section for the (d,a) reaction on the $^{9}$Be nucleus is mainly due to the mechanism of 5He cluster transfer at backward scattering angles. However, [13] concluded that the nuclear reaction $^{9}$Be(d, a)7Li goes through the mechanism the compound nucleus. One of the main arguments in favor of this mechanism is that the energy of the nuclear reaction at 12 and 14 MeV lies precisely in the range of giant resonance in the excitation spectrum of the 11B compound nucleus. In addition, in the work [12], theoretical calculations for the cross section (d, a) of the reaction underestimate the experimental data at the front scattering angles, which suggests the presence of two more direct mechanisms: sequential transfer of the n + p system. Nevertheless, one more experiment should be carried out, and the data should be analyzed in a different energy range. It is assumed that on the basis of a new set of experiments in the framework of the proposed Project, it will be possible to answer the question which mechanism has the greatest contribution to the reaction cross section $^{9}$Be(d,a)7Li.
In Ref. [14 - 16] theoretically studied the reactions of elastic and inelastic scattering, as well as single-nucleon transfers in the interactions of d and 3,4He and with the $^{9}$Be nucleus. The interaction potential of light d and 3,4He particles with $^{9}$Be was calculated in the framework of the folding model using the wave function of the $^{9}$Be ground state in the three-particle 2$\alpha$ + n approximation. Within the framework of the coupled channel method and the distorted wave Born approximaton method, the differential cross sections of inelastic scattering and the reactions of single nucleon transfer were calculated using the folding potential. The obtained angular distributions are in good agreement with experimental data. Thus, it was concluded that the three-cluster approximation $^{9}$Be = 2$\alpha$ + n constructed on the basis of the multicluster dynamic model taking into account the Pauli principle, provides an adequate description of the internal structure and properties of the $^{9}$Be nucleus.  
In particular, interesting information about the properties of exotic nuclei is manifestation of their cluster structure, which can also be obtained from experiments on elastic scattering. The study of exotic states in light nuclei is a priority in the development of nuclear physics in recent decades. Exotic conditions in most cases have a rarefied structure and increased size. Of particular interest is the second excited $^{12}$C state, the Hoyle state, which in many models has a cluster structure and increased dimensions [17, 18]. A similar increase is predicted for states in nuclei 11B and 13C [19, 20], possible analogues of the Hoyle state. As mentioned above, in the framework of the $\alpha$-condensate theory, the radius of the Hoyle state is 1.4–1.7 times the radius of the ground state and the existence of states which has a radius close to the radius of the uranium nucleus are predicted in the $^{12}$C and 11B nuclei.
Hoyle's state plays an important astrophysical role in the synthesis of $^{12}$C in the Universe. The formation of elements heavier than carbon goes through this state. If it would not exist, the rate of carbon formation reaction was 7 orders of magnitude lower. This state is interesting from the hypothesis describing it as a gas of interacting alpha particles, which can be represented as a Bose condensate. Despite numerous studies, the properties of this state, lying above the threshold of decay into three alpha particles, are still poorly understood.
To study the theory of nuclear reactions and its use to describe experimental data a method the Distorted Wave Born Approximation (DWBA) has been successfully applied. The distorted wave method is a good tool for describing nuclear reactions with products registered in the ground state. For transfer channels with excited states, calculations using the DWBA method in many cases give disagreement with the experimental data. In this case, the disagreement is explained by the fact that the wave function of the bound state for the output channel corresponds to a discrete eigenvalue, while the state of this excited nucleus in the experimental data has a resonance with a certain width in the order of several MeV. An adequate theoretical analysis of such problems requires other approaches and methods that require taking into account the degrees of freedom associated with the dynamics of the movement of three clusters and its influence on the mechanisms of nuclear reactions.
The consideration of the continuum in the processes of nuclear reactions is used in the Continuum Discretized Coupled Channels (CDCC) approach [21]. According to the name of the method, the essence of the method consists in discretizing the continuum and in coupling the reaction channels. As in the DWBA method, CDCC requires data which take into account the internal structure of colliding nuclei. The structure is determined by calculations of spectroscopic amplitudes, which strongly depend on the theoretical models under consideration. To date, a well-developed approach for studying the internal structure of the light nuclei of the ground state is ab initio. The ab initio method, which uses realistic NN forces, differs from the shell model in independence from model spaces. There is another method developed on the basis of the CDCC method, this is XCDCC [22]. The addition to the name “X” means taking into account the excited state of the target nucleus. Recently, a continuum discretization method was proposed based on imaginary correlated gaussian functions [23]. This approach successfully describes the displacement phases of elastic scattering of the p +  system and reproduces well the real Coulomb functions.
The aim of this project are: an adequate description of the properties and characteristics of $^{9}$Be and $^{12}$C nuclei using the above methods, obtaining new experimental data on the scattering of light particles by $^{9}$Be and $^{12}$C nuclei, extraction of experimental differential cross sections for elastic, inelastic, single-nucleon and cluster transmission channels. It is expected that the features of the experimental methodology, the latest equipment and modern theoretical approaches will provide an opportunity to better understand the properties and characteristics of the studied nuclei.
Research on this topic is one of the rapidly developing areas of modern nuclear physics of all major scientific centers of the world. Kazakhstan scientists also conduct intensive research in these areas, and work closely with renowned scientists, major research centers, and the results are quite competitive internationally. This is evidenced by the scientific publications of Project managers in international journals with a high impact factor, and thereby show the high importance of this project. New data on the cross sections of nuclear reactions, reliable parameters of the optical potential, and spectroscopic characteristics of the excited states of 5He, 8Be, 10,11,12B, and $^{12}$C nuclei will be useful for testing various cluster nuclear models and for carrying out model calculations of nucleosynthesis reactions for astrophysical and thermonuclear applications.
The project is expected to obtain a number of new and relevant results. The new experimental technique will allow, firstly, to conduct experiments at high energies, and secondly, to experimentally measure cross sections at large scattering angles. In the first case, it is important to note that in the scientific world there are still no enough data on the scattering of 3He and $\alpha$ particles by $^{9}$Be nuclei above energies of 35 MeV. It is very important to conduct an experiment at such energies, since they make it possible to measure the cross sections of relatively heavy reaction products, which reflect direct evidence of cluster transfer. In the second case, modern devices provide a great opportunity to extract information from nuclear reactions at the backward scattering angles. The available information from world-famous databases on the cross section for the 3He + $^{9}$Be reactions for today, unfortunately, has limitations due to poor resolution of detectors. Therefore, the realization of this project, which is not only of academic interest, but also of great practical importance, is highly desirable.

\label{Chapter1} % For referencing the chapter elsewhere, use \ref{Chapter1} 

%----------------------------------------------------------------------------------------

% Define some commands to keep the formatting separated from the content 
\newcommand{\keyword}[1]{\textbf{#1}}
\newcommand{\tabhead}[1]{\textbf{#1}}
\newcommand{\code}[1]{\texttt{#1}}
\newcommand{\file}[1]{\texttt{\bfseries#1}}
\newcommand{\option}[1]{\texttt{\itshape#1}}

%----------------------------------------------------------------------------------------

\section{Nuclear reactions}

\subsection{Elastic scattering}
\subsection{Inelastic scattering}
\subsection{Transfer reactions}

\section{Mechanisms of nuclear reactions}
\section{Cluster phenomenon}


