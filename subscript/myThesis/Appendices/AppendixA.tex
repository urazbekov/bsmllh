% Appendix A

\chapter{Definitions from the quantum theory of angular momenta} % Main appendix title

\label{AppendixA} % For referencing this appendix elsewhere, use \ref{AppendixA}
In this Appendix some definitions used in this work and the table integrals are presented.

$\bullet~$ A total angular momentum $\mathbf{j}$ are decomposed into two angular momenta $\mathbf{j}_1$ and $\mathbf{j}_2$ by means of the Clebsch-Gordan coefficient. For example, to quote a basis $\vert ~ jm \rangle $ with the angular momentum $ j$ with its $z$-component $m$, the Clebsch-Gordan coefficient can be represented as follow
\begin{equation}
\label{the_CG_coefficient}
\vert ~ jm \rangle =\sum_{m_1 m_2} \langle ~ j_1 m_1~j_2 m_2~ \vert ~j m~  \rangle ~ \vert ~j_1 m_1 \rangle~ \vert ~j_2 m_2 \rangle,
\end{equation}
For non-zero values of the coefficient (\ref{the_CG_coefficient}) vectors $\mathbf{j}_1$, $\mathbf{j}_2$ and $\mathbf{j}$ must satisfy the rule of triangle:
\begin{align*}
\vert j_1 - j_2 \vert \leq j \leq j_1 + j_2 \\
\vert j - j_2 \vert \leq j_1 \leq j + j_2 \\ 
\vert j_1 - j \vert \leq j_2 \leq j_1 + j 
\end{align*}
and the condition
\begin{equation*}
m=m_1+m_2.
\end{equation*}


$\bullet~$  If there are three vectors $\mathbf{j}_1, \mathbf{j}_2$ and $\mathbf{j}_3$, one can get a total angular momentum $\bf j$ in two ways

\begin{align}
\label{6j_basis1}
\bf j & =\bf \left( j_1 + j_2 \right) + j_3 = j_{12} + j_3 \\
\label{6j_basis2}		
& = \bf j_1 + \left( j_2  + j_3 \right) = j_1 +j_{23}
\end{align}

The Basis $ \vert (j_1 j_2)j_{12},j_3; jm \rangle$ and the basis $\vert j_1,(j_2 j_3); ~jm \rangle$ corresponding to Eq.~(\ref{6j_basis1}) and Eq.~(\ref{6j_basis2}) are related through a factor $U(~j_1 j_2 j j_3;~ j_{12} j_{23})$, which is the Racah coefficient:
\begin{equation}
\vert (j_1 j_2)j_{12},j_3; jm \rangle = \sum_{j_{23}} U(~j_1 j_2 j j_3;~ j_{12} j_{23}) ~ \vert j_1,(j_2 j_3); ~jm \rangle.
\label{racah_U}
\end{equation}

Four angular momenta, $\bf j_1,~j_2,~j_3$ and $\bf j_4$, are added into the total momentum $\bf j$ by
\begin{align}
\label{9j_basis1}
\bf j & =\bf \left( j_1 + j_2 \right) + ( j_3 + j_4) = j_{12} + j_{34} \\
\label{9j_basis2}		
& = \bf ( j_1 + j_3 ) + \left( j_2  + j_4 \right) = j_{13} +j_{24}
\end{align}

$\bullet~$  Two basis  $\vert~ j_1 j_2 (j_{12}),~j_3 j_4 (j_{34});~jm \rangle$ and $\vert~ j_1 j_3 (j_{13}),~j_2 j_4 (j_{24});~jm \rangle $, constructed respectively on the scheme Eq.~\ref{9j_basis1} and Eq.~\ref{9j_basis2},  are related as follow
\begin{align}
\label{9j}
\vert~ j_1 j_2 (j_{12}),~j_3 j_4 (j_{34});~jm \rangle = \sum_{j_{13},j_{24}} 
\begin{bmatrix}
j_1 & j_2 & j_{12} \\ 
j_3 & j_4 & j_{34} \\ 
j_{13} & j_{24} & j
\end{bmatrix} 
\vert~ j_1 j_3 (j_{13}),~j_2 j_4 (j_{24});~jm \rangle 
\end{align}
where transformation coefficient with square brackets is called a unitary 9j-symbol.


$\bullet~$   A spacial spherical harmonics is expressed like
  \begin{equation}
  \mathcal{Y}_{lm}({\bf r}) = r^{l} Y_{lm}(\hat{r})
  \end{equation}
  where $Y_{lm}(\hat{r})$  -- spherical function, which is a eigenfunction of angular part the $\Delta_{\hat{r}}$ Laplace operator.
  For ${\bf r} = a{\bf r}_1+b {\bf r}_2$ a decomposition of the spacial spherical harmonics $\mathcal{Y}_{lm}({\bf r})$ leads to the following equality
  \begin{align}
  \mathcal{Y}_{lm}({\bf r} = a{\bf r}_1+b {\bf r}_2)=& \sum_{l_1,l_2,m_1,m_2} a^{l_1} b^{l_2}
  \langle ~ l_1 m_1~l_2 m_2~ \vert ~l m~  \rangle \mathcal{D}(l,l_1,l_2) \times \nonumber \\
   & \times \mathcal{Y}_{l_l m_1}({\bf r_1})  \mathcal{Y}_{l_2 m_2}({\bf r_2}) \nonumber \\
   = & \sum_{l_1,l_2} a^{l_1} b^{l_2}
   \mathcal{D}(l,l_1,l_2) \left[ \mathcal{Y}_{l_l}({\bf r_1}) \times \mathcal{Y}_{l_2}({\bf r_2}) \right]_{lm} 
  \end{align}
  with the condition $l=l_1+l_2$, and $\mathcal{D}(l,l_1,l_2) $ is given by
  \begin{equation}
  \label{sphHarmDecomp}
  \mathcal{D}(l,l_1,l_2)  = \sqrt{\frac{4 \pi (2l+1)!}{(2l_1+1)! (2l_2+1)!}}
  \end{equation}
$\bullet~$  Spherical harmonics with the momenta $l_1$ and $l_2$ are coupled as follow
\begin{equation}
\left[ Y_{l_1}(\hat{r}) \times Y_{l_2}(\hat{r})\right]_{lm} =  \mathcal{C}(l_1,l_2,l) Y_{lm}(\hat{r})
\end{equation}
where the $\mathcal{C}(l_1,l_2,l)$ coefficient reads as
\begin{equation}
\mathcal{C}(l_1,l_2,l) = \sqrt{\frac{(2l_1+1)(2l_2+1)}{4 \pi (2l+1)}} \langle ~ l_1 0~l_2 0~ \vert ~l 0~  \rangle
\end{equation}

$\bullet~$ It would be useful also note a coupling between two spherical hyper harmonics kind of
\begin{equation}
\left[ Y^{(l_1l_2)}_{l_{12}}(\hat{{\bf r}}_1,\hat{{\bf r}}_2) \times Y^{(l_3l_4)}_{l_{34}}(\hat{{\bf r}}_1,\hat{{ \bf r}}_2) \right]_{lm}= \sum_{l_{13}l_{24}} {E}^{l_1l_2l_{12}l_2l_4l_{34}l}_{l_{13}l_{24}}  Y^{(l_{13}l_{24})}_{lm}(\hat{{\bf r}}_1,\hat{{\bf r}}_2)
\end{equation}
where the coupling coefficient ${E}^{l_1l_2l_{12}l_2l_4l_{34}l}_{l_{13}l_{24}}$ is given as
\begin{equation}
\label{hyperSphTrans}
{E}^{l_1l_2l_{12}l_2l_4l_{34}l}_{l_{13}l_{24}} = 
\begin{bmatrix}
l_1 & l_2 & l_{12} \\ 
l_3 & l_4 & l_{34} \\ 
l_{13} & l_{24} & l
\end{bmatrix}
\mathcal{C}(l_1,l_3,l_{13})\mathcal{C}(l_2,l_4,l_{24}).
\end{equation}

$\bullet~$ The modified spherical Bessel function may be expressed as follows
\begin{equation}
i_l(x)=\sqrt{\frac{\pi}{2 x}} I_{l+\tfrac{1}{2}}(x)
=\sum_{k=0}^{\infty} \frac{x^{2k+l}}{(2k)!!(2k+2l+1)!!},
\label{mod_sph_bessel}
\end{equation} 
where $I_{l}(x)$ is a modified Bessel function of the first kind.

$\bullet~$ The integral depending on the parameters $\lambda$ and $\alpha$ 
\begin{equation*}
\int_0^\infty dx~ x^{\lambda} 
\exp \left( - \alpha x^2 \right)
\end{equation*}
can be expressed in explicit form as follows
\begin{equation}
\mathcal{I} \left( \lambda,\alpha \right)=
 2^{1+\lambda}\frac{\Gamma \left( 1+\lambda \right)  }{ \left( \alpha \right) ^{1+\lambda}} .
\label{table_integral_1}
\end{equation}
where, $\Gamma(x)$ -- the Gamma function.

$\bullet~$  An integral kind of
\begin{equation*}
\int_0^\infty dx~ x^{2n+l+2} 
\exp \left(- \tfrac{1}{2} v x^2 \right)
i_{l}(|w|y)
\end{equation*}
 has analytical form as
\begin{equation}
\mathcal{I}(n,~l,~v,~|w|) = 
\frac{\pi}{2}\frac{(2n)!! |w|^l}{v^{n+l+3/2}}
\exp\left( \frac{w^2}{2v} \right)
L^{l+1/2}_{n} \left( -\frac{w^2}{2v}\right)
\label{table_integral_2}
\end{equation} 
 where, $L_n^{l+1/2}(x)$ -- is the associated Laguerre polynomial.

Integral with six parameters  
\begin{equation}
\int_0^\infty dx \int_0^\infty dy~
x^{2\lambda+l+2} y^{2n+l+2} 
\exp\left(-\tfrac{1}{2} \alpha x^2-\tfrac{1}{2} \beta y^2 \right)
i_{l}(|\rho|xy)
\end{equation}
may be expressed as follow
\begin{align}
\mathcal{I}(\lambda, l, n, \alpha, \beta, |\rho|) =& \sqrt{\frac{\pi}{8}}(2l)!!~ \Gamma(l+n+\tfrac{3}{2})~|\rho|^{n} ~\beta^{-l-n-\tfrac{3}{2}} \times \label{table_integral3} \\
& \times \sum_{\kappa=0}^{l} \frac{\Gamma(\kappa+\lambda+n+\tfrac{3}{2})}{\kappa! (l-\kappa)! \Gamma(\kappa+n+\tfrac{3}{2})}
\left(\frac{\rho^2}{2\beta}\right)^{\kappa} \left( \frac{\alpha}{2} - \frac{\rho^2}{2\beta}  \right)^{-\kappa-\lambda-n-\tfrac{3}{2}}.
 \nonumber
\end{align}